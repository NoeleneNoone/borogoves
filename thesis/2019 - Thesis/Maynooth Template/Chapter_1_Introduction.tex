
% this file is called up by thesis.tex
% content in this file will be fed into the main document

%: ----------------------- introduction file header -----------------------
\chapter{Introduction}\label{ch:intro}

\begin{textsl}
{\small In this chapter, we discuss the motivations behind the work of this thesis and provide an overview of the material presented in the following chapters.}
\end{textsl}

\vspace*{1cm}

% ----------------------------------------------------------------------
%: ----------------------- introduction content ----------------------- 
% ----------------------------------------------------------------------
\section{Motivation}
Before the advent of the Cloud and rapid software release models, software was developed and delivered to customers via a method called Waterfall. Waterfall is a non-iterative design process used in software development. One of the disadvantages of Waterfall was the time taken from development, test and delivery to the customer (i.e. a release every 12 to 36 months).

With the broad adoption of consumer computing in the 1990s, customer demand precipitated a move away from Waterfall to a leaner ``little and often'' approach. Agile development introduced the concept that a team should have a feature or component ready for release at periodic stages during its development cycle. The adoption of agile development practices allowed for incremental releases to made at a much faster rate.

A refinement of agile came about with the round-the-clock availability of Cloud computing and services. Cloud computing adopted the practice of using a network of remote servers to store, manage, and process data, rather than a local server or a personal computer. With the ``always available'' paradigm, new development practices were required. A new development model known as continuous delivery (CD) was created. CD differs from agile development in that development teams develop software that is ready for release at any time during development.

Delivering software for the Cloud represents a challenge for both micro teams, Small Medium Enterprises (SMEs) and startups, in part due to the rapid release methods (i.e. CD) adopted and the numerous ways in which software defects can be detected. 

Likewise, as applications are hosted on a Cloud-based infrastructure, production outages (critical software defects) can occur in a variety of ways due to the complex nature of distributed computing.

Instant messaging is a popular form of real-time communication between groups. Adoption of such collaboration tools took off during the mid-1990s with tools such as AOL Instant Messenger, ICQ and PowWow and IRC. As businesses realised the potential of real-time communication, additional corporate offerings were released: IBM Sametime, MSN and Yahoo!. 

The core attraction for businesses was chiefly the ability for teams regardless of size or location to collaborate on a wide range of topics (e.g. Cloud outage remediation). However, with the growth of next-generation solutions such as Slack, IBM Workspace and Microsoft teams, large volumes of text data is generated. Making sense of this data can be a challenge to teams, given the lack of inbuilt analytical tooling.

We extend previous work by studying field defect, production outage events and real-time collaboration data. Through empirical research on a myriad of enterprise and open-source datasets, we provide a series of frameworks that can be used to turn endless streams of data into high-value information.

\section{Overview}
This thesis is divided into the following chapters:

\textbf{Chapter 1: Introduction}

This is the introductory chapter and broad discussion of the context that, in part motivates this proposed work.

\textbf{Chapter 2: Literature Review}

For the literature review chapter, background and related research are reviewed and discussed.

\textbf{Chapter 3: Social Testing}

This chapter presents two studies of software testing conducted as part of the release of a Cloud-based enterprise application. The concept of social testing is introduced, and we demonstrate how this idea can be used to reduce field defects. We pay particular attention as to why this type of study is importation for small teams. 

\textbf{Chapter 4: Outage Modelling}

This chapter discloses details of a study conducted into Cloud outages, using an enterprise dataset. Both outage inter-arrival and services are modelled. We extend the idea of the critical defect discovery to Cloud outages. These events have the potential to disrupt a Cloud service for a temporary or extended period of time.

\textbf{Chapter 5: Outage Simulation}
In this chapter, we take the inter-arrival and service times from our Cloud outage study and simulate the arrival of future outages events. From this simulation staffing requirements to manage such events can be inferred. 

\textbf{Chapter 6: Chat Discourse Modelling}

This chapter presents a framework that can be used to model real-time chat discourse using both parametric and non-parametric methods. Collaboration applications are used to diagnose Cloud outage events. Modelling conversation can serve as an analogue to Cloud outage service times.

\textbf{Chapter 7: Chat Discourse Segmentation and Boundary Identification}

This chapter describes a novel technique to segment chat conversations to provide an improved degree of understanding for topic modelling. We also consider a text classification framework to identify chat conversation boundaries. Text mining of chat discourse can be useful to understand key terms in prior Cloud outage engagements. Identification of conversation boundaries may provide more targeted topic modelling on a per discussion basis.

\textbf{Chapter 8: Conclusions}

This last chapter summarises the key findings, learning outcomes and pathways for future work.

\section{Publications} 

Over the lifetime of my research, the following conference papers and journal have been peer-reviewed and published. I have grouped the papers by thesis chapter for ease of reference.

\textbf{Chapter 3: Social Testing}
 
\begin{itemize}
  \item Social testing: A framework to support adoption of continuous delivery by Small Medium Enterprises. (CSCESM 2015: The Second International Conference on Computer Science, Computer Engineering, \& Social Media)
  \item Social dogfood: A framework to minimise Cloud field defects through crowdsourced testing. (28th Irish Signals and Systems Conference - 2017)
\end{itemize}


\textbf{Chapter 4: Outage Modelling}

\begin{itemize}
  \item Of queues and cures: A solution to modelling the inter time arrivals of cloud outage events (36th Conference on Applied Statistics in Ireland - 2016)
  \item Are you being served: A Framework to manage Cloud outage repair times for Small Medium Enterprises. (27th Irish Signals and Systems Conference - 2016)
\end{itemize}


\textbf{Chapter 5: Outage Simulation}

\begin{itemize}
  \item Obscured by the cloud: A resource allocation framework to model cloud outage events (The Journal of Systems and Software (Elsevier) - 2017)
\end{itemize}


\textbf{Chapter 6: Chat Discourse Modelling}

\begin{itemize}
  \item Different every time: A framework to model real-time instant message conversations. (21st Finnish-Russian University Cooperation in Telecommunications conference - 2017)
\end{itemize}


\textbf{Chapter 7: Chat Discourse Segmentation and Boundary Identification}

\begin{itemize}
  \item Bundles: A framework to optimise topic analysis in real-time chat discourse (????? - 2018)
  \item Hello and Goodbye: A framework to identify conversation boundaries in real-time chat discourse (????? - 2018)
\end{itemize}

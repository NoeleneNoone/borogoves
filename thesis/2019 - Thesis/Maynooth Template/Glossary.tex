% content in this file will be fed into the main document

% Glossary entries are defined with the command \nomenclature{1}{2}
% 1 = Entry name, e.g. abbreviation; 2 = Explanation
% You can place all explanations in this separate file or declare them in the middle of the text. Either way they will be collected in the glossary.

% required to print nomenclature name to page header
%\markboth{\MakeUppercase{\nomname}}{\MakeUppercase{\nomname}}


% ----------------------- contents from here ------------------------
\chapter{Glossary}

%\addnotation $\alpha$: {some variable that means something to me}{alpha}
\begin{longtable}{ p{4cm} p{0.7\linewidth}}
\textbf{Term} & \textbf{Meaning}  \\ 
\\


Agile development & Agile development describes a set of principles for software development whereby requirement evolve through collaboration with cross-functional teams. Agile encourages an iterative development style. Agile promotes the idea that early delivery can be achieved at regular intervals throughout the product development lifecycle. First examples of iterative development can be traced back to the late 1950s however it was not until the idea of rapid software delivery gained wider adoption \cite{martin1991rapid}. \\

Business Support System (BSS) & Herbert Simon states that the BSS is the ``connection point'' between external relations (customer s, partners and suppliers) and an enterprise's products and services \cite{simon2009empirically}. \\

Continuous delivery & Continuous Delivery or (CD) is the ability to get changes of all types -- including new features, configuration changes, bug fixes and experiments -- into production, or into the hands of users, safely and quickly in a sustainable way \cite{humble2010continuous}.\\

DevOps & Is a practice that highlights the collaboration between software development and infrastructure personnel. DevOps may also refer to a team which has a core function to build, deploy and maintain a Cloud infrastructure. These definitions are referenced from some sources such as ``What is DevOps?'' by Mike Loukides \cite{loukides2012devops} and ``The Phoenix Project'' by Kim et al. \cite{kim2014phoenix}. \\

Downtime (Outage) & The Alliance for Telecommunications Industry Solutions (ATIS) define the term downtime as periods when a system is unavailable. Downtime or outage duration refers to a period that a system fails to provide or perform its primary function \cite{Outdefine}. \\

Field defects &  Refers to all defects found by customers using the software product post-release \cite{ieee1998ieee}. \\

Functional Testing & Testing which is focused on the specified functional requirements and does not verify the interactions of system functions \cite{ieee1998ieee}.\\

Maintenance window &  In information technology and systems management, a maintenance window is a period designated in advance by the technical staff, during which preventive maintenance that could disrupt service may be performed. This definition is also provided by ATIS \cite{Outdefine}.  \\

Micro team & Deshpande defined a Micro team as one that typically consists of three or four persons, with just a single or at most two developers, a business analyst and a project manager or surrogate customer \cite{deshpande2011study}. \\

Performance Testing & In software engineering Performance testing is performed to determine how a system performs regarding responsiveness and stability under a particular workload \cite{ieee1998ieee}.\\

Queuing theory & Is the study of events that form waiting lines or queues. In queuing theory, a model is constructed so that queue lengths, inter-arrival and service times can be predicted \cite{kleinrock1975queuing, gross2008fundamentals, sundarapandian2009probability}. \\

SME & Enterprise Ireland define a small enterprise as an enterprise that has fewer than 50 employees and has either an annual turnover and an annual balance sheet total not exceeding \euro10m. A medium enterprise is an enterprise that has between 50 employees and 249 employees and has either an annual turnover not exceeding \euro50m or an annual balance sheet total not exceeding \euro43m \cite{SMEdefine}.\\

Startup & Croll and Yoskovitz in their book lean analytics state that a startup is an organisation formed to search for a repeatable and scalable business model \cite{croll2013lean}.\\

System Testing & Testing conducted on a complete integrated system to evaluate the system's compliance with its specified requirements.  System test, unlike Functional testing, validates end-to-end system operations within the broader environmental context. Therefore system testing should be conducted on an environment, which closely mimics's customer behaviour\cite{ieee1998ieee}.\\

Tiger Team & The term tiger team was first introduced in 1964. Dempsey et al. defined a tiger team as a team of undomesticated and uninhibited technical specialists, selected for their experience, energy, and imagination, and assigned to track down relentlessly every possible source of failure in a spacecraft subsystem \cite{dempsey1964program}. \\

Waterfall model & A waterfall model is a non-iterative design process used in software development. The design process works through some phases (cascading like a waterfall) as follows: conception, initiation, analysis, design, construction, testing, production/implementation and maintenance. The term developed by Herbert D. Benington in 1956 \cite{benington1983production}.\\

\end{longtable}
\newpage
